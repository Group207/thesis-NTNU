
% From https://www.overleaf.com/learn/latex/Glossaries

\makeglossaries % Prepare for adding glossary entries


\newglossaryentry{latex}
{
        name=latex,
        description={Is a mark up language specially suited for
scientific documents}
}

\newglossaryentry{bibliography}
{
        name=bibliography,
        plural=bibliographies,
        description={A list of the books referred to in a scholarly work,
typically printed as an appendix}
}

\newglossaryentry{maths}
{
    name=mathematics,
    description={Mathematics is what mathematicians do}
}

\newglossaryentry{desksim}
{
    name=DeskSim,
    description={The current simulator used by Lokførerskolen today}
}

\newglossaryentry{lokforerskolen}
{
    name=Lokførerskolen,
    description={The Norwegian Train Driver Academy, a vocational school educating train drivers, and the client of this project}
}

\newglossaryentry{scrum}
{
    name=Scrum,
    description={An agile framework for project management within software development}
}

\newglossaryentry{kanban}
{
    name=Kanban,
    description={A framework for agile software development which focuses on balancing demands with available capacity}
}

\newglossaryentry{sprint}
{
    name=sprint,
    description={A short period where a team aims to complete a set amount of work withing the Scrum methodology}
}

\newglossaryentry{jira}
{
    name=Jira,
    description={A web service for tracking and managing product development, developed by Atlassian}
}

\newglossaryentry{jmonkeyengine}
{
    name=jMonkeyEngine,
    description={A 3D game engine written in Java by jME Core Team in 2003}
}

\newglossaryentry{unity}
{
    name=Unity,
    description={A cross-platform game engine developed by Unity Technologies}
}

\newglossaryentry{unrealengine}
{
    name=Unreal Engine,
    description={A 3D game and computer graphics engine developed by Epic Games}
}

\newglossaryentry{godot}
{
    name=Godot,
    description={An open source, cross-platform game engine}
}

\newglossaryentry{cryengine}
{
    name=CryEngine,
    description={A 3D game engine developed by Crytek}
}

\newglossaryentry{open3dengine}
{
    name=Open 3D Engine,
    description={An open source 3D game engine managed by The Linux Foundation}
}

\newglossaryentry{opensource}
{
    name=open source,
    description={Computer software released publicly and freely for anyone to modify, use and/or publish}
}

\newglossaryentry{crossplatform}
{
    name=cross platform,
    description={The practice of developing the same software for multiple technologies or environments}
}

\newglossaryentry{github}
{
    name=GitHub,
    description={A service for hosting software repositories and version control using Git}
}

\newglossaryentry{garbagecollection}
{
    name=garbage collection,
    description={When the compiler frees up memory by deleting unused or obsolete  memory references}
}

\newglossaryentry{gizmo}
{
    name=gizmo,
    description={An overlay with a functional purpose, providing visually relevant options to the user}
}

\newglossaryentry{mesh}
{
    name=mesh,
    description={A graphic primitive, represented as geometric shapes when rendered}
}

\newglossaryentry{viewport}
{
    name=viewport,
    description={The area on the screen visible to the user}
}

\newglossaryentry{blueprint}
{
    name=blueprint,
    description={Unreal Engine's own node-based programming language for visual scripting}
}

\newglossaryentry{heads-updisplay}
{
    name=heads-up display,
    description={Elements overlaying the screen, displaying information to the user}
}

\newglossaryentry{spline}
{
    name=spline,
    description={A mathematical function for interpolating smoothly between multiple points, creating a customizable curve}
}

\newglossaryentry{actor}
{
    name=actor,
    description={A base class for any object that can be placed into a level in Unreal Engine}
}

\newglossaryentry{instanced dynamic material}
{
    name=instanced dynamic material,
    description={A material which is instance-editable and can be edited during runtime}
}

\newglossaryentry{emissive light}
{
    name=emissive light,
    description={The model itself can act as a light source, without needing a separate light source}
}

\newglossaryentry{packaging}
{
    name=packaging,
    description={Unreal Engine's process of collecting files and resources and assembling them into an executable software}
}

\newglossaryentry{visualstudio}
{
    name=Visual Studio,
    description={}
}

\newglossaryentry{solution}
{
    name=solution,
    description={}
}

\newglossaryentry{doxygen}
{
    name=Doxygen,
    description={Standard tool for generating documentation from \cpp projects}
}

% --------------------
% ----- Acronyms -----
% --------------------


\newacronym{ertms}{ERTMS}{European Rail Traffic Management System}
\newacronym{vr}{VR}{Virtual Reality}
\newacronym{xr}{XR}{Extended Reality}
\newacronym{ar}{AR}{Augmented Reality}
\newacronym{ntnu}{NTNU}{Norges teknisk-naturvitenskapelige universitet}
\newacronym{dmi}{DMI}{Driver-Machine Interface}
\newacronym{3d}{3D}{Three Dimensional}


\newacronym{ue}{UE}{Unreal Engine}
\newacronym{hud}{HUD}{Heads-Up Display}
\newacronym{ui}{UI}{User Interface}
\newacronym{ai}{AI}{Artificial Intelligence}
\newacronym{api}{API}{Application Programming Interface}
\newacronym{bp}{BP}{Blueprint}
\newacronym{cpu}{CPU}{Central Processing Unit}
\newacronym{mvp}{MVP}{Minimum Viable Product}
\newacronym{https}{HTTPS}{HyperText Transfer Protocol Secure}
\newacronym{html}{HTML}{HyperText Markup Language}
\newacronym{json}{JSON}{JavaScript Object Notation}
\newacronym{jwt}{JWT}{JSON Web Token}
\newacronym{ide}{IDE}{Integrated Development Environment}
\newacronym{mit}{MIT}{Massachusetts Institute of Technology}