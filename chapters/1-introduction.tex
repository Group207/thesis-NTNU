\chapter{Introduction} % John Ole
\raggedbottom % allowing space

\section{Background}

The Norwegian Train Driver Academy, \textit{\Gls{lokforerskolen}}, is a public vocational school part of The Norwegian Railway Directorate, educating locomotive drivers over the course of a year. Their study program consists of both a theoretical part as well as physical on field work, where they follow an experienced locomotive driver and gets to try driving and operating a train. As a part of their theoretical part they use a train simulator called \Gls{desksim}. It was originally implemented as an extra tool for students to learn the new signal system \acrshort{ertms} when it first arrived, but is also being used to simulate real life scenarios especially high risk situation you usually do not see often \cite{lokforerskolen}. Since then they also implemented a \acrshort{vr} mode where you can simulate and train on switching tracks and connecting different things to trains.

\section{Problem Area}

As a part of their education they use \Gls{desksim}, a self developed in-house train simulator. \Gls{desksim} is built upon \textit{\Gls{jmonkeyengine}}, a Java-based game engine developed in 2003 that has recently become outdated in some areas. To avoid irrelevancy, Lokførerskolen started their long-term project of changing the game engine from \textit{\Gls{jmonkeyengine}} into a more modern one.

The project is a combination of two parts. The first part is an analysis and comparison of different modern game engines. The second part of the project is to create a demo in the chosen game engine.

\section{Delimitation}

In this project we compared a few different game engines for developing a simulator. While we were going to compare different engines, we did not perform a comprehensive technical analysis of each engine and its features. We also limited ourselves to about 4-5 different engines. 

Due to the technical nature of the simulator, we only implemented core functionality related to movement of the trains, as well as a basic signaling system. The goal of the project was therefore not to provide a realistic, detailed, physics-based simulation, but a tool that imitates realistic train movement. For this reason the movement of the train is not physics-based. The signal implementation has basic logic for controlling it's statuses.  

As the task was not to develop new content for the simulator, the group reused assets from the existing simulators when possible. Improving the graphical fidelity was not a goal of the project, but was done as a part of using a newer, more capable engine. 

\section{Target Audience}

\subsection{Product}
The main audience of the product is people using the simulator. This is people connected to \Gls{lokforerskolen}, such as students and teachers. To use the product you would need a device like a computer that can run the simulator, as well as an account to login into the simulator. 

\subsection{Thesis}

The target audience for the report is people that want insight into our development process from a project plan to deployment. The main target audience for this thesis is the development team and administrators at \Gls{lokforerskolen}, as well people interested in game development, simulator applications, project management and software engineering. The thesis, more specifically the game engine analysis can be of interest to developers in the process of choosing a game engine. The thesis is written in context of computer programming, and presumes the reader has basic knowledge about computers and software development. 

\section{Group Background}
% Why we chose this
All members of the group are students at \acrshort{ntnu} in Gjøvik. 

% bachelor i programmering, tidligere fag (grunnleggende c++, cloud, mobile, game-programming, ai, graphics)
\todo{det som står i kommentarer - Thomas}
% interest in game development? 

\section{Project Goals}
The project will provide \Gls{lokforerskolen} with our recommendation of game engine for their train simulator and develop a demo in our chosen engine that they can continue to build upon. 

\subsection{Main Goals}
We were given a list by the client of various features they wanted us to implement in the demo, and we managed to develop all of the main features including all of the optional features as well.

The main goal is to create a demo scenario with the following features:
\begin{itemize}
    \item Must include at least one train, two signals, one train-\acrshort{dmi}, a train track and a simple landscape.
    \item The train must be able to move using the controllers \Gls{lokforerskolen} uses today.
    \item The train must follow the railway in a realistic way.
    \item Signals must be able to change colors based on specific events happening in game.
\end{itemize}

\subsection{Part Goals}
\begin{itemize}
    \item Create a tool for placing, editing and deleting \acrshort{3d} models in the game world, such as trains, buildings or signals.
    \item Make it possible to save the world when it is edited.
    \item Create a tool for creating, editing and deleting train tracks.
    \item Train tracks must be obtain curves and not only go in a straight line.
    \item Make it possible to place a train on the tracks and drive it.
\end{itemize}

\section{Analysis Goals}
Look into what game engines are available and compare and analyze them against each other based on criteria given to us by \Gls{lokforerskolen}: 
\begin{itemize}
    \item Must be a modern game engine
    \item Must support functionality for virtual reality
    \item Must be capable of reproducing all functionality of \Gls{desksim}
    \item The game engine should be easy to learn
    \item The game engine should be able to reuse existing 3D assets from \Gls{desksim}
\end{itemize}

\section{Constraints}

Because we developed the application for \Gls{lokforerskolen}, they had some imposed constraints and boundaries we had to follow. Any deviations had to be discussed and clarified with the client first.

Since we didn't have any previous experience or knowledge to train operation, we were not responsible for the educational content of the simulator. Although \Gls{lokforerskolen} did teach us the basics they were not a requirement set by them, and we were therefore only responsible for the development of the product.

We were only supposed to develop a demonstration of how some of their functionalities from \Gls{desksim} would function in a different engine. As such we were not tasked to develop all of the functionalities in \Gls{desksim} into our own simulator, but only a few key features set by \Gls{lokforerskolen}.

% Tid 20 mai
% 17. januar - 20. mai

\todo{Skriv om tidsfrist - Endre}

\section{Roles}
\textbf{Thomas Arinesalingam} was the \textit{Project Leader}. His role specific responsibilities were to ensure that all group members had equal right to express their thoughts. Be the project's "man of action", motivating the development. He ensured that all submissions were delivered on time. 

\textbf{John Ole Bjerke} was the \textit{Research Manager}, overseeing all research and ensuring the level of obtained knowledge is adequate before the development starts.

\textbf{Endre Heksum} was the \textit{\Gls{scrum} Master} for the project and was responsible for the development of the product and took the role of sprint leader.

\textbf{Henrik Markengbakken Karlsen} was the \textit{Writer of Minutes}, writing minutes from all meetings and making sure all team members logs both time and work log.

\textbf{Tom Røise} was our supervisor during the project, and provided guidance and academic support to the group during the development process.

\textbf{Isak Kvalvaag Torgersen} represented \Gls{lokforerskolen} who was our client which we developed the train simulator for. 


\section{The Report}

The thesis is written using latex and is based on a template provided by \acrshort{ntnu}[kilde]. It is divided into eleven chapters:
\todo{Lett til kilde for ntnu latex}
\begin{enumerate}
    \item \textbf{Introduction} contains an introduction to the thesis. 
    \item \textbf{Choice of Engine} contains the analysis of game engines.
    \item \textbf{Requirements} contains all of the requirements formed for the development of DeskSim v2.
    \item \textbf{Development} contains a description of the project's development plan and process.
    \item \textbf{Technical Design} contains the system, application and network architecture of the product.
    \item \textbf{Product Overview} contains a description of the product seen from a user perspective.
    \item \textbf{Implementation} contains information, code and design explanations to some of the functionality present in DeskSim v2.
    \item \textbf{Deployment} contains the required information on how to set up the project, packaging and releasing it and how to deploy the documentation.
    \item \textbf{Testing} contains the user tests and system tests performed on the system.
    \item \textbf{Discussion} contains a reflection regarding the game engine analysis, the project process and the final result.
    \item \textbf{Conclusion} Summarizes the final result of the product and contains some final words.
\end{enumerate}

\todo{Write an introduction about the report, what and how it represents information - Henrik}
% Organization, terminology, practical (layout, style and fonts etc...)
% - 
% How is the report setup

% game engine analysis vs development and demo - time usage and focus

% general info about report

